\documentclass[11pt,a4paper]{article} 

%\usepackage[utf8]{inputenc}
\usepackage{graphicx}
\usepackage{hyperref}
\usepackage[margin=2cm]{geometry}
\usepackage{lineno}
\usepackage{booktabs}
\usepackage{array}
\usepackage[labelfont=bf, font=small]{caption}
\usepackage[onehalfspacing]{setspace}
\usepackage{floatrow}
\usepackage{xr}

\DeclareFloatFont{small}{\scriptsize}% "scriptsize" is defined by floatrow, "tiny" not
\floatsetup[table]{font=small}
\linenumbers

\title{Twister3: a simple and fast microelectrode twister}
\author{Jonathan P. Newman, Jakob Voigts, Matthew A. Wilson}

\begin{document}

\maketitle

\begin{abstract}
    \noindent
    We present Twister3, a twisted wire probe-making machine. This device
    greatly increases the speed and repeatability of constructing twisted wire
    probes (TWPs; e.g. stereotrodes and tetrodes) compared to existing options.
    It is cheap, well documented, and all associated designs and source code
    are open-source. Twister3 is of interest to any lab using TWP recordings in
    freely moving animals, for example, in conjunction with microdrive arrays
    or static implants.
\end{abstract}

\section{Introduction}
Since their introduction \cite{}, twisted-wire probes (TWPs; e.g stereotrodes
\cite{} and tetrodes \cite{}) have been a reliable method for obtaining
multiple single-unit extracellular spiking data in freely moving animals. They
are cheap (~xx USD/cm), small enough to cause minimal inflammation \cite{},
biocompatible enough to be used over many months \cite{}, and mechanically
flexible, such that they move with along with neural tissue, rather than as a
rigid, cantilevered beam, which mitigates "drift" \cite{} and improve long term
stability \cite{}. Their contacts are close (~20 um) and therefore allow much
improved unit separability compared to single wire probes \cite{}. Although the
introduction of modern silicon \cite{} and carbon-fiber probes \cite{} offer
major advances in terms of channel density and size, respectively, and even
rival microwires in terms of mechanical flexibility \cite{}, TWPs will remain a
ubiquitous recording method due to their simplicity, performance, and low cost.

Although quite simple to make \cite{}, constructing TWPs is a tedious and
time-consuming process. For modern, easy to assemble microdrive designs \cite{},
making TWPs are a rate-limiting step. Typically, TWPs are are
made in three actions (Fig. \ref{f:manual-steps}):

\begin{enumerate}
    \item \textbf{Electrode Folding}: tungsten or nickel-chromium (`nichrome')
        resistance wire is drawn from a spool and folded by hand 2, 4, or 8
        times to make stereo-, tet-, or oct-trodes, respectively.
    \item \textbf{Twisting}: the folded wire bundle is draped over a smooth
        metal rod. The free end of the bundle is loosely linked a motor armature using a
        magnetic
        \footnote{\url{https://neuralynx.com/hardware/tetrode-spinner-2.0}} or
        mechanical coupling
        \footnote{\url{http://www.open-ephys.org/twister}}.
        The motor then turns the wire bundle into a helix.
    \item \textbf{Wiring fusing}: A hot air gun is used to fuse the insulation
        on the wire bundle, forming a rigid, implantable unit.
\end{enumerate}

\begin{figure}
    \centering
    \includegraphics[width=0.25\textwidth]{standin.jpg}
    \caption{TODO: Manual tetrode making steps.}
    \label{f:manual-steps}
\end{figure}

Recently,
SpikeGadgets\footnote{\url{http://www.spikegadgets.com/main/home.html}}
introduced a tetrode twisting machine that provides an arrangement of four
pre-wound wire bobbins that allow the user to draw a multi-wire bundle without
the folding step. Because folding is the most time consuming step of the
tetrode making process, this device improves the speed at which tetrodes could
be created. Additionally, this method minimizes wire handling, which is
beneficial because wire ends up in direct contact with neural tissue. Despite
these advantages, the cost (10,000 USD) and complexity of this device has
hampered its adoption. Inspired by the SpikeGadgets design, we have created an
open-source twisting machine that is approximately 10x times cheaper and
faster. Our device replaces a servo motor with high speed stepping
motor and modern micro-stepping driver in order to increase twisting
speed while maintaining precise control motor acceleration and smooth motor
actuation. A new counter-balanced spring system provides constant tension on
tetrode wire during twisting. Our design allows a moderately trained ($\sim$1 hr of
experience with device) operator to make $\sim$100 TWPs per hour.

Here we present descriptions of how this device works, materials and assembly
information, driver electronics designs, and descriptions of control firmware.
In addition we provide detailed usage instructions, an exploration of twisting
parameters, and show data obtained with tetrodes made using this machine. All
designs and source code associated with this project can be found on its git
repository\footnote{\url{https://github.com/jonnew/twister3}}.

\section{Notable design elements}

\section{3D-printed leaf spring and for rapid turn rates}
In order to achieve straight TWPs, wire must be twisted together while under
tension. In existing designs \cite{}, a weight is hung from the wire bundle and
a motor is loosely coupled to the weight in a way that does not constrain axial
motion. This provides constant tension on the wire bundle due to gravity and
allows while allowing it to decrease in length as it is twisted. An variant of
this method uses magnets hold the weight in place, but, in our experience, this
unnecessary and is prone to causing wire breakage due to the nonlinear
force/distance magnetic relationship.

A key design criteria of our device was that TWPs need to be turned extremely
quickly. We aimed for < 10 seconds of actual twisting time. Assuming 100 total
revolutions (typical for tetrodes) this translates to an average turn rate of
600 RPM. Because previous methods rely on loose motor coupling, they were
unsuitable to meet our speed requirements. The high centripetal forces involved
in rapid turning inevitably lead to instability of the coupling mechanism,
causing the bundle to vibrate wildly. Therefore, we sought a way to rigidly
constrain the motor and wire bundle in the turning plane, but still maintain
freedom in the axial direction to allow bundle shortening during the turning
process. To meet this goal, we made use of selective laser-sintered polyether
block amide (PEBA) to create a monolithic, combined leaf-spring/wire-clip
mechanism (Fig. XX). PEBA provides rubber-like mechanical qualities allowing
us to achieve a spring with an spring-constant low enough for use with tetrode
wire ($\sim$68 mN/mm in the relevant range of motion). 

The wire bundle interfaces with this mechanism using a standard alligator clip
that has been coated in shrink wrap, which we have found effective for holding
the microwire bundle. A strong, neodymium magnet provides the final linkage
between the leaf spring and this alligator clip.

\section{Quick-draw wire feeder} 
To further increase the speed at which TWPs can be made, microwire needs to be
drawn, and attached to the  motor quickly, and ideally with minimal separate
actions being taken by the operator. To facilitate the rapid draw of wire from
stock spools, we designed a torsional-spring based feeding element that allows
wire to be drawn rapidly from stock spools but applies enough friction to
counter the increase in wire tension during a twist, transferring all slack
compensation to the leaf spring. The holding force of this mechanism is
adjustable. We have found that the lowest available setting, which results in a
stiction threshold force of $\sim$XX N per spool, is adequate to counter wire
tension and therefore prevent spools from feeding during twisting.

\section{Advanced Motor control hardware and control firmware smooth turning}

\section{Usage}

\subsection{Making TWPs}


\subsection{Loading Bobbins}


\subsection{Using the controller}

\begin{figure}
    \centering
    \includegraphics[width=\textwidth]{../resources/control-box/control-box-usage.png}
    \caption{Twister3 control box.}
    \label{f:ctrl-box}
\end{figure}

\section{Materials and Assembly}

\subsection{Control Electronics}
The control board is comprised of the following blocks: power regulation,
motor-driver, microcontroller, and user interface. Wherever possible, we used
pre-assembled modules in order to ease assembly and increase ease of soldering.

%\externaldocument{twister3}
\setlength{\tabcolsep}{3pt}

\begin{table}[!htbp]
\centering
\caption{Control box (Fig.~\ref{f:ctrl-box}) bill of materials. A continuously
    updated bill of materials is located on
    \href{https://docs.google.com/spreadsheets/d/1tdc3wfE6V87q8yqBOQvDj7WylAztkh6_2kuL-YzyB0g/edit\#gid=0}{this
    Google sheet}. Table entries labeled ``common'' are ubiquitous and
    therefore listing specific part numbers is futile due to rapid supplier
    turn over. PCB design files are located at this
    \href{https://github.com/jonnew/twister3/tree/master/control-board}{link}.
    The PCB is split into two halves separated by a breakaway v-cut. One half
    serves as the top panel of the enclosure, visible in (Fig.~\ref{f:ctrl-box}
    (B)), and the other has all electronics and is mounted beneath the front
panel.}
\label{t:elec_bom}

\begin{tabular}{lllllll}
\multicolumn{6}{l}{\textit{\textbf{Control Box Bill of Materials}}} \\
\toprule
\textbf{Qty} &
\textbf{Schematic Name} &
\textbf{Schematic Value} &
\textbf{Description} &
\textbf{Supplier} &
\textbf{Supplier Part No.} &
\textbf{Manufacturer Part No.} \\
\midrule

% AUTOGENERATED
1 & U5 & N/A & Teensy 3.2 & PRJC & & \\
& & & & & & \\
3 & M1, M2, M3 & N/A & M3 Screws, 10 mm & McMaster & 92000A120 & \\
& & & & & & \\
1 & N/A & N/A & 24V Wall Adapter & Amazon & & \\
& & & & & & \\
1 & S1 & EC12P & Rotary Encoder & Sparkfun & 10982 & EC12P \\
1 & S1 & N/A & Knob & Sparkfun & 10597 & P-15x13-T-T2-24A \\
1 & U6 & GDM1602K & LCD Display 5V & Sparkfun & 709 & GDM1602K \\
1 & N/A & N/A & Hot-Air Station & Sparkfun & 14557 & \\
& & & & & & \\
1 & For U5 & Teensy Header & 2x7 Header & Digikey & 609-3485-1-ND & 61001421121 \\
2 & For U5 & Teensy Header & 1x14 Header & Digikey & 732-5325-ND & 61301411121 \\
1 & For U5 & Teensy Socket & 2x7 Socket & Digikey & S7110-ND & PPPC072LFBN-RC \\
2 & For U5 & Teensy Socket & 1x14 Socket & Digikey & S7012-ND & PPTC141LFBN-RC \\
1 & For U6 & LCD Header & 1x16 Header & Digikey & 732-5327-ND & 61301611121 \\
1 & For U6 & LCD Socket & 1x16 Socket & Digikey & S7014-ND & PPTC161LFBN-RC \\
2 & J10, J11 & Stepper Stick Header & 1x8 Header & Digikey & 732-5321-ND & 61300811121 \\
2 & J10, J11 & Stepper Stick Socket & 1x8 Socket & Digikey & S7006-ND & PPTC081LFBN-RC \\
5 & J1, J8, J9 & N/A & 1x4 Socket & Digikey & S7002-ND & PPTC041LFBN-RC \\
3 & J3, J4, J5 & N/A & 1x4 SMD Header & Digikey & SAM12225-ND & TSM-104-02-L-SV \\
2 & For U2 & N/A & Board Spacers & Digikey & 952-3465-ND & R6010-00 \\
3 & M4, M5, M6 & N/A & M3 SMD Nuts & Digikey & 732-10918-1-ND & 9775056360R \\
1 & J2 & N/A & Screw Terminal & Digikey & ED10563-ND & OSTVN04A150 \\
1 & J1 & PJ-063BH & Barrel Jack & Digikey & CP-063BH-ND & PJ-063BH \\
4 & C2, C6, C7, C8 & 10 nF, 50 V & Capacitor, X7R & Digikey & 478-1227-1-ND & 06035C103KAT2A \\
1 & C16 & 470 nF, 10 V & Capacitor, X7R & Digikey & 1763-1190-1-ND & 0402BW474K160YT \\
2 & C1, C3 & 1 uF, 50 V & Capacitor, X7R & Digikey & 587-3247-1-ND & UMK107AB7105KA-T \\
1 & C4 & 15 uF & Tantalum Capacitor & Digikey & 399-3775-1-ND & T491D156K035AT \\
1 & C5 & 220 uF & Tantalum Capacitor & Digikey & 399-9744-1-ND & T491D227M010AT \\
1 & R6 & 470 \si{\ohm}& Resistor, Thick film & Digikey & Common & Common \\
2 & R1, R9, R10, R11, R12 & 4.7 k\si{\ohm} & Resistor, Thick film & Digikey & Common & Common \\
5 & R2, R3, R4, R5, R17 & 10 k\si{\ohm} & Resistor, Thick film & Digikey & Common & Common \\
1 & R13 & 20 k\si{\ohm} & Resistor, Thick film & Digikey & Common & Common \\
2 & R8, R16 & 20 k\si{\ohm} & Trim Pot & Digikey & SM-42TX203CT-ND & SM-42TX203 \\
1 & L1 & 68 uH & Inductor & Digikey & 513-1386-1-ND & DR125-680-R \\
1 & D1 & DB2W60400L & Schottky Diode & Digikey & P15184CT-ND & DB2W60400L \\
1 & U1 & IS31FL3193 & LED Driver & Digikey & 706-1215-1-ND & IS31FL3193-DLS2-TR \\
1 & U3 & MAX6816 & Debouncer & Digikey & MAX6816EUS+TCT-ND & MAX6816EUS+T \\
1 & U2 & MIC4680-5.0YM-TR & DC DC Converter & Digikey & 576-1221-1-ND & MIC4680-5.0YM-TR \\
1 & U4 & TMC2100 & Silent Stepper Stick & Digikey & 1460-1159-ND & N/A \\
1 & N/A & N/A & Enclosure & Digikey & 36-705-ND & 705 \\

\bottomrule
\end{tabular}
\end{table}


\section{Twisting Parameters}
- Back twisting vs forward only
- Height of bobbins vs coil
- Good parameters

\medskip

\bibliographystyle{unsrt} % Used BibTeX style is unsrt
\bibliography{refs}

\end{document}
