\documentclass[11pt,a4paper]{article}

\usepackage{booktabs}
\usepackage[T1]{fontenc}
\usepackage{graphicx}
\usepackage{hyperref}
\usepackage[margin=2cm]{geometry}
\usepackage{lineno}
\usepackage{booktabs}
\usepackage{array}
\usepackage[labelfont=bf, font=small]{caption}
\usepackage[onehalfspacing]{setspace}
\usepackage{floatrow}
\usepackage{xr}
\usepackage{siunitx}
\usepackage{enumitem}
\usepackage{wasysym}

\DeclareFloatFont{small}{\scriptsize} % "scriptsize" is defined by floatrow
\floatsetup[table]{font=small}
\linenumbers

\title{Twister3: a simple and fast microelectrode twister}
\author{Jonathan P. Newman, Jakob Voigts, Mattias Karlsson?, \& Matthew A. Wilson}

\begin{document}

\maketitle

\begin{abstract}
    \noindent
    We present Twister3, a microwire twisting machine. This device greatly
    increases the speed and repeatability of constructing twisted microwire
    probes (e.g.\ stereotrodes and tetrodes) compared to existing options. It is
    cheap, well documented, and all associated designs and source code are
    open-source. Twister3 is of interest to any lab performing twisted
    microwire recordings in freely moving animals, for example, in conjunction
    with microdrive arrays or static implants.
\end{abstract}

\tableofcontents

\section{Introduction}
Since their introduction~\cite{}, twisted microwire probes (TWPs; e.g
stereotrodes~\cite{} and tetrodes~\cite{}) have been a reliable method for
obtaining multiple single-unit extracellular spiking data in freely moving
animals. They are cheap ($\sim$5--10 USD/m using Sandvik PX000004), small enough
to cause minimal inflammation~\cite{}, biocompatible enough to be used over
many months~\cite{}, and mechanically flexible, such that they move with along
with neural tissue, rather than behaving like a rigid, skull-coupled beam. This
greatly mitigates \``drift''~\cite{} and improve long term stability and
stability~\cite{} compared to with contemporary low-density silicon probes
\url{}. Their contacts are close (~20 um) and therefore allow much improved
unit separability compared to single wire probes~\cite{}. Although the
introduction of modern silicon~\cite{} and carbon-fiber probes~\cite{} offer
major advances in terms of channel density and size, respectively, and rival
microwires in terms of mechanical flexibility~\cite{}, TWPs will remain a
ubiquitous recording method for the foreseeable future due to their simplicity,
good performance, and low cost.

Although simple to make~\cite{}, constructing TWPs is a tedious and
time-consuming process. For modern, easy to assemble microdrive designs
\cite{}, making TWPs is often a rate-limiting step. Typically, TWPs are are
made in three actions~\cite{}:

\begin{enumerate}
    \item \textbf{Folding}: tungsten or nickel-chromium (`nichrome')
        resistance wire is drawn from a spool and folded by hand 2 or 4
        times to make stereo- or tetrodes, respectively.
    \item \textbf{Twisting}: the folded wire bundle is draped over a smooth
        metal rod. The free end of the bundle is loosely linked a motor
        armature using a mechanical\footnote{\url{http://www.open-ephys.org/twister}}
        or magnetic\footnote{\url{https://neuralynx.com/hardware/tetrode-spinner-2.0}}
        coupling.  The motor then twists the wire bundle into a helix.
    \item \textbf{Fusing}: A hot air gun is used to fuse the insulation
        on the wire bundle, forming a rigid, implantable probe.
\end{enumerate}


Recently,
SpikeGadgets\footnote{\url{http://www.spikegadgets.com/main/home.html}}
introduced a tetrode twisting machine that provides an arrangement of four
pre-wound wire bobbins that allow the user to draw a multi-wire bundle without
the folding
step\footnote{\url{http://www.spikegadgets.com/downloads/Tetrode\%20Machine\%20Manual.pdf}}.
Because folding is the most time consuming step of the tetrode making process,
this device greatly improves the speed at which tetrodes can be created.
Additionally, this method minimizes human wire handling, which is beneficial
because wire ends up in direct contact with neural tissue. Despite these
advantages, the cost (10,000 USD) and complexity of this device has hampered
its adoption. Inspired by the SpikeGadgets design, we have created an
open-source twisting machine that is approximately 10x times cheaper and faster
(Fig.~\ref{f:design-features} (A)). Our device replaces a servo motor with high
speed stepping motor and modern micro-stepping driver in order to increase
twisting speed while maintaining precise control motor acceleration and smooth
motor actuation. A new counter-balanced spring system provides constant tension
on tetrode wire during twisting. Our design allows a moderately trained
($\sim$1 hr of experience with device) operator to make $\sim$100 TWPs per
hour.

\begin{figure}
\centering \includegraphics[width=1.0\textwidth]{./fig/design-features.png}
\caption{\textbf{(A)} Overview of the Twister3 mechanical components. The
    motor and wire feeder assemblies are used to rapidly construct TWPs by
    drawing wire from the feeder, clipping it to the motor, and executing a
    twist. The motor, wire guide, and stock spool assemblies are used to
    load wire onto the bobbins in the feeder after they are depleted.
    \textbf{(B)} Right: Wire tensioner mechanism. Right: 3D printed
    leaf spring showing deformation under tension. The shape of the spring
    permits approximately vertical deformation of the wire attachment point
    so that the center axis is maintained as the bundle is shorted due to
    twisting. \textbf{(C)} Right: Wire quick draw mechanism. Left: Isolated
    single bobbin indicating the wire tension, due to the leaf spring in (A),
    and counter stiction due to the adjustable torsional spring.}
\label{f:design-features}
\end{figure}

Here we present descriptions of how this device works, materials and assembly
information, driver electronics designs, and descriptions of control firmware.
In addition we provide detailed usage instructions, an exploration of twisting
parameters, and show data obtained with tetrodes made using this machine. All
designs and source code associated with this project can be found on its git
repository\footnote{\url{https://github.com/jonnew/twister3}}.

\section{Notable design elements}

\subsection{3D-printed leaf spring for rapid turn rates}
In order to achieve straight TWPs, wire must be twisted together while under
tension. In existing designs~\cite{}, a weight is hung from the wire bundle and
a motor is loosely coupled to the weight in a way that does not constrain axial
motion. This provides constant tension on the wire bundle due to gravity while
allowing it to decrease in length as it is twisted. A variant of this method
uses magnets hold the weight in place, but, in our experience, this unnecessary
and is prone to causing wire breakage due to the nonlinear force/distance
magnetic relationship.

A key design criterion for our device was that TWPs need to be turned very
quickly. We aimed for <10 seconds of twisting time (time when motor is in
motion) per TWP. Assuming 100 total revolutions (typical for tetrodes) this translates
to an average turn rate of 600 RPM.\@ Because previous methods rely on loose
motor coupling (with the exception of the aforementioned SpikeGadgets design),
they were unsuitable to meet our speed requirements. The high centripetal
forces involved in rapid turning inevitably lead to instability of the coupling
mechanism, causing the bundle to vibrate wildly. Therefore, we sought a way to
rigidly constrain the motor and wire bundle in the turning plane, but still
maintain freedom in the axial direction to allow bundle shortening during the
turning process. To meet this goal, we made use of selective laser-sintered
polyether block amide (PEBA) to create a monolithic, combined leaf
spring/wire-clip mechanism (Fig.~\ref{f:design-features} (B)). PEBA provides
rubber-like mechanical qualities resulting in a spring-constant low enough for
use with tetrode wire ($\sim$68 mN/mm in the relevant range of motion TODO:
confirm this).

The wire bundle interfaces with this mechanism using a standard alligator clip
that has been coated in shrink wrap, which we have found effective for holding
the microwire bundle. A strong, neodymium magnet provides the final linkage
between the leaf spring and this alligator clip (Fig.~\ref{f:design-features}
(A, B)).

\subsection{Quick-draw wire feeder}
To further increase the speed at which TWPs can be made, microwire needs to be
drawn and attached to the motor quickly. Ideally, this process should occur
with as few separate motor actions being taken by the operator as possible. To
facilitate the rapid draw of wire from stock spools, we designed a
torsional-spring based feeding assembly that allows wire to be drawn rapidly
from stock feeder bobbins (Fig.~\ref{f:design-features} (C)). This mechanism
applies enough friction to feeder bobbins to counter increased wire tension
during a twist, transferring all slack compensation to the leaf spring
(Fig.~\ref{f:design-features} (B)). The holding force of this mechanism is
adjustable to account for the elastic deformability of different wire
materials. We have found that the lowest available setting, and working with
standard tetrode nichrome wire \url{}, the lowest stiction threshold force
setting of $\sim$XX N per spool, is adequate to counter wire tension and
therefore prevent spools from feeding during twisting.

\subsection{Advanced motor control hardware for smooth turning} 
To obtain precise control over motor acceleration, speed, and position we opted
to use a bipolar stepper motor to perform wire twisting. We have found that,
due to their discretized motion, stepper motors can vibrate resonantly with
taught microwire, resulting in irregular twists and wire damage. To overcome
this issue, we drive our motor using an advanced microstepping driver
(Fig.~\ref{f:ctrl-box} (A));
\href{https://www.trinamic.com/products/integrated-circuits/details/tmc2100/}{Trinamic
TMC2130}). This driver has internal circuitry which is capable of generating up
to 256 microsteps/step by interpolating over the step commands provided by the
host microcontroller. In our case, we use 200 steps/revolution (\ang{1.8})
motor. Microstep commands from the microcontroller are provided at 16
microsteps/step, which are further interpolated to 256 microsteps/step by
internal driver circuitry. This results in a motor update resolution of 3,200
microsteps/revolution (\ang{0.113}), and a motion discretization of 51,200
steps/revolution (\ang{0.007}). Therefore, the motor operates approximately as
smoothly as a continuous DC motor but with much improved motion control
dynamics.

\begin{figure}
\centering
\includegraphics[width=\textwidth]{./fig/electronics-blocks.png}
\caption{\textbf{(A)} Control electronics block diagram. All user IO is
    provided via a combined rotary encoder and button. The XX's programmable
    interrupt timer (PIT) controls step command to the motor driver, independent of
    nominal operation. \textbf{(B)} Control box.}
\label{f:ctrl-box}
\end{figure}

We used an Arduino-compatible Teensy
LC\footnote{\url{https://www.pjrc.com/teensy/teensyLC.html}} microcontroller
module perform step timing calculation. Its ARM Cortex-M0+ includes an
integrated programmable interrupt timer (PIT) which is used to provide
jitter-free step commands to the motor driver while floating-point acceleration
calculations are
performed\footnote{\url{https://github.com/luni64/TeensyStep}}.

\section{Usage}

\subsection{Using the control box}
The control box (Fig.~\ref{f:ctrl-box} (B)) is powered using a 12VDC,
center-positive barrel jack that supplies at least 1.5A. It has a single user
input: a control knob consisting of a combined quadrature rotary encoder and
tactile push-button. This knob permits the following user actions:

\begin{itemize}[noitemsep]
    \item \textbf{Press:} cycle through different settings (forward turns,
        backward turns, turn speed, turning mode)
    \item \textbf{Turn:} increment or decrement the selected setting
        depending on turn direction.
    \item \textbf{Press and hold:} execute the turn sequence using the
        current settings
    \item \textbf{Press during motion:} cancel the twist and stop the motor
        immediately.
\end{itemize}

The control box is used to perform two tasks: twisting electrodes (turn mode 0)
and loading bobbins with microwire (turn mode 1). The turn mode is selected and
changed using the dial on the controller. The selected turn mode is
shown in the upper right corner of the liquid crystal display (LCD). After a
mode is selected, all turning parameters (speed, forward and backwards turns)
pertain to that mode only. All parameters are stored in non-volatile memory
when a turn is started by pressing and holding the control knob. In the
following sections, we detail how to use the mechanical components for making
TWPs and loading bobbins with stock wire.

\subsection{Loading Bobbins}
Before twisting electrodes, the bobbins on the wire feeder assembly must be
loaded with microwire. To load the bobbins, follow these steps:

\begin{enumerate}[noitemsep]
    \item Remove the wire shield by removing its M6-retention screw
    \item Remove one set of bobbins by unscrewing the M3 bold that serves as
        the axle.
    \item Take the bobbins and spacers off the axle.
    \item Remove any remaining microwire from bobbin and make sure it is clean
        of dirt and debris.
    \item Remove the twisting attachment from the motor.
    \item Place a bobbin on the motor using the embedded magnets
    \item Adjust the position of the wire guide such that the pointy part
        points directly into the center of the wire groove on the bobbin. The
        tip of the wire guide should be a few mm away from the bobbin.
    \item Place the stock spool bearings into the stocks pole and mount on the
        steel post using the M3 bolt
    \item Feed the tetrode wire from the stock spool through the wire guide and
        wrap once around the bobbin.
    \item Set the controller to `mode 1'
    \item Select the desired loading speed. We have found that 100 RPM works
        well.
    \item Select the desired number of turns to load the bobbin. The
        circumference of the bobbin is $\sim$10 cm. The length of wire loaded
        on the bobbin is therefore $turns \times 10\ cm$.
    \item Start the turn and wait until it is finished. \textit{Be careful not
        touch moving parts during this process}: the microwire needs to have
        constant tension to ensure it is properly loaded on the bobbin.
    \item Repeat the process for the remaining bobbins.
    \item Put the bobbins back on their axle on the wire feeder assembly. Loose
        wire ends should point inward on both sides of the assembly.
    \item Replace the wire shield.
\end{enumerate}


\subsection{Making TWPs}
To twist tetrodes, use the following steps. Note that if you are interested in
makeing stero- or tritrodes, simply don't load two or one bobbins, respectively.

\begin{enumerate}[noitemsep]
    \item Pull down the wire from all of the bobbins and group the tips with
        your fingers. If one wire is much longer than the others, trim it.
    \item Clamp the tip of bundle with the alligator clip.
    \item Pull up on the twisting attachment's leaf spring up until under
        slight tension, about 1 cm.
    \item Move the alligator clip down to meet the magnet on the twisting attachment.
        Now, the leaf spring should be in opposition with the upward force
        produced by item the tetrode wire. Each of the wires should be pulled
        straight. If any wire has item slack, its bobbin can be turned
        backwards slightly until it is taught.
    \item Make sure the controller is in `mode 0` and has been set to the
        desired number of turns and turn speed.
    \item Press and hold the knob down to perform a twist.
    \item When finished, fuse wires \textit{starting from point at which
        they separate towards the bobbins} using the hot air gun at $\sim$480 deg.
        C. We have found that fusing form the bottom will cause the lower
        portion of the TWP to `absorb' slack from above resulting in a very
        fine twist pitch and an TWP that is shorter than intended.
    \item Roll each of the bobbins forward a bit in order to release tension
        on the tetrode wire.
    \item Cut the tetrode wire above the point at which the wires are fused
    \item Pull the alligator clip off the magnet and cut the finished
        tetrode into a storage box.
\end{enumerate}

\subsection{Twisting Parameters}
- Back twisting vs forward only
- Height of bobbins vs coil
- Good parameters

\section{Materials and Assembly}

\subsection{Mechanical Components}
The mechanical portion of Twister3 consists of standard hardware, co-opted
optomechanical components, and 3D-printed parts. Wherever possible, we used
standard (and easy to replaceable) components. The mechanical bill of materials
is shown in Table~\ref{t:mech_bom}.

%\externaldocument{twister3}
\setlength{\tabcolsep}{3pt}

\begin{table}[!htbp]
\centering
\label{t:mech-bom}
\caption{Mechanical\ bill of materials. An continuously updated bill of
    materials is located on
    \href{https://docs.google.com/spreadsheets/d/1tdc3wfE6V87q8yqBOQvDj7WylAztkh6_2kuL-YzyB0g/edit\#gid=0}{this
    google sheet}.}

\begin{tabular}{llll}
\multicolumn{4}{l}{\textit{\textbf{Mechanical Bill of Materials}}} \\
\toprule
\textbf{Qty} &
\textbf{Description} &
\textbf{Supplier} &
\textbf{Supplier Part No.} \\
\midrule

% AUTOGENERATED
1 & NEMA-17 Stepper Motor & Digikey & 1460-1163-ND \\
& & & \\
10 & 3 $\times$ 7 $\times$ 3mm Bearings & Boca Bearing & MR73-2RSC \\
& & & \\
1 & M3 $\times$ 4mm Set screw & McMaster-Carr & 92605A098 \\
2 & M3 $\times$ 6mm Standoff & McMaster-Carr & 94868A162 \\
1 & M3 $\times$ 0.5 mm Thread Size, 500 mm Long & McMaster-Carr & 90024A219 \\
2 & M3 $\times$ 0.5 mm Thread, 60 mm Long & McMaster-Carr & 92290A131 \\
8 & M6 $\times$ 1 mm Thread, 14 mm Long & McMaster-Carr & 92855A615 \\
2 & M6 $\times$ 1 mm Thread, 25 mm Long & McMaster-Carr & 9271K578 \\
1 & Dowel Pin 3mm Diameter, 6mm Long & McMaster-Carr & 93600A267 \\
4 & Music-Wire Steel Torsion Spring & McMaster-Carr & 9271K21 \\
2 & Zinc-Plated, M3 $\times$ 0.5 mm Thread & McMaster-Carr & 90695A033 \\
2 & Dowel Pin 3/16" Diameter, 2-1/2" Long & McMaster-Carr & 98380A520 \\
& & & \\
2 & \diameter1.5" Mounting Post, M6 Taps, L = 250 mm & ThorLabs & P250/M \\
2 & Swivel-Base Mini-Post Holder, 2" (51 mm) Tall, 1/4" (M6) Slot & ThorLabs & MSL2 \\
2 & Mini-Series Optical Post, \diameter6 mm, L = 50 mm & ThorLabs & MS2R/M \\
1 & Aluminum Breadboard, 300 mm x 600 mm x 12.7 mm, M6 Taps & ThorLabs & MB3060/M \\
1 & \diameter1.5" Post Mounting Clamp, 112.5 mm x 112.5 mm, Metric & ThorLabs & C1545/M \\
& & & \\
10 & Neodymium Block Magnets, 1/8" cube & KJ Magnetics & B222G-N52 \\
2 & Neodymium Block Magnets, 1/2" x 1/4" x 1/16" & KJ Magnetics & B841-N52 \\

\bottomrule
\end{tabular}
\end{table}


Aside from standard mechanical components, several 3D-printed parts are
required. The 3D-printing bill of materials in shown in Table~\ref{t:3dbom}.
These parts are available for direct  purchase from third-party 3D-printing
services via the links provide in the table.

%\externaldocument{twister3}
\setlength{\tabcolsep}{3pt}

\begin{table}[!htbp]
\centering
\caption{3D-printing bill of materials. An continuously updated bill of
    materials is located on
    \href{https://docs.google.com/spreadsheets/d/1tdc3wfE6V87q8yqBOQvDj7WylAztkh6_2kuL-YzyB0g/edit\#gid=0}{this
    google sheet}. The right-most column provides links to order parts from each supplier.}
\label{t:3d-bom}

\begin{tabular}{llllll}
\multicolumn{4}{l}{\textit{\textbf{3D-Printing Bill of Materials}}} \\
\toprule
\textbf{Qty} &
\textbf{Description} &
\textbf{Material} &
\textbf{Supplier} &
\textbf{STL File in Repository} &
\textbf{Link} \\
\midrule

% AUTOGENERATED
1 & Feeder Base & Nylon PA12 & Shapeways & \verb|twister3_arm-base.STL| & \url{http:/shpws.me/QtfT} \\
1 & Motor Mount & Nylon PA12 & Shapeways & \verb|twister3_motor-mount.STL| & \url{http:/shpws.me/QtfL} \\
1 & Bobbin Shield & ``Fine detail plastic'' & Shapeways & \verb|twister3_bobbin-shield.STL| & \url{http:/shpws.me/QtfU}\\
    2 & Bobbin Spacer & ``Fine detail plastic'' & Shapeways & \verb|twister3_bobbin-spacer.STL| & \url{http:/shpws.me/QtfR} \\
    2 & Bobbin Washer & ``Fine detail plastic'' & Shapeways & \verb|twister3_bobbin-washer.STL| & \url{http:/shpws.me/QtfP} \\
    4 & Bobbin & ``Fine detail plastic'' & Shapeways & \verb|twister3_bobin.STL| & \url{http:/shpws.me/QtfM} \\
    1 & Spring Rotor Base & ``Fine detail plastic'' & Shapeways & \verb|twister3_spring_rotor_base.STL| & \url{http:/shpws.me/QtfK} \\
    1 & Wire Guide & ``Fine detail plastic'' & Shapeways & \verb|twister3_wire-guide-base.STL| & \url{http:/shpws.me/QtfI} \\
    2 & Stock Spool Bearing Case & ``Fine detail plastic'' & Shapeways & \verb|twister3_stock-spool-bearing-case.STL| & \url{http:/shpws.me/QUtz} \\
    1 & Twist Wire Aligner & ``Fine detail plastic'' & Shapeways & \verb|twister3_twist-aligner.STL| & \url{http://shpws.me/R6mo} \\
    1 & Spring Rotor Top & PEBA 2301 & Sculpteo & \verb|twister3_spring-top.STL| & \url{https://bit.ly/2H32cna} \\

\bottomrule
\end{tabular}
\end{table}


\subsection{Control Electronics}
The control board is comprised of the following blocks: power regulation,
motor-driver, microcontroller, and user interface (Fig.~\ref{f:ctrl-box} (A)).
Wherever possible, we used pre-assembled modules (microcontroller, motor
driver, and LCD display). The bill of materials for this board is shown in
Table~\ref{t:elec_bom}. PCBs desings and gerber files are available on the
repository\footnote{\url{https://github.com/jonnew/twister3/tree/master/control-board}}.

%\externaldocument{twister3}
\setlength{\tabcolsep}{3pt}

\begin{table}[!htbp]
\centering
\caption{Control box (Fig.~\ref{f:ctrl_box}) bill of materials. A continuously
    updated bill of materials is located on
    \href{https://docs.google.com/spreadsheets/d/1tdc3wfE6V87q8yqBOQvDj7WylAztkh6_2kuL-YzyB0g/edit\#gid=0}{this
    Google sheet}. Table entries labeled ``common'' are ubiquitous and
    therefore listing specific part numbers is futile due to rapid supplier
    turn over. PCB design files are located at this
    \href{https://github.com/jonnew/twister3/tree/master/control-board}{link}.
    The PCB is split into two halves separated by a breakaway v-cut. One half
    serves as the top panel of the enclosure, visible in (Fig.~\ref{f:ctrl_box}
    (B)), and the other has all electronics and is mounted beneath the front
panel.}
\label{t:elec_bom}

\begin{tabular}{lllllll}
\multicolumn{6}{l}{\textit{\textbf{Control Box Bill of Materials}}} \\
\toprule
\textbf{Qty} &
\textbf{Schematic Name} &
\textbf{Schematic Value} &
\textbf{Description} &
\textbf{Supplier} &
\textbf{Supplier Part No.} &
\textbf{Manufacturer Part No.} \\
\midrule

% AUTOGENERATED
1 & U5 & N/A & Teensy 3.2 & PRJC & & \\
& & & & & & \\
3 & M1, M2, M3 & N/A & M3 Screws, 10 mm & McMaster & 92000A120 & \\
& & & & & & \\
1 & N/A & N/A & 24V Wall Adapter & Amazon & & \\
& & & & & & \\
1 & S1 & EC12P & Rotary Encoder & Sparkfun & 10982 & EC12P \\
1 & S1 & N/A & Knob & Sparkfun & 10597 & P-15x13-T-T2-24A \\
1 & U6 & GDM1602K & LCD Display 5V & Sparkfun & 709 & GDM1602K \\
1 & N/A & N/A & Hot-Air Station & Sparkfun & 14557 & \\
& & & & & & \\
1 & For U5 & Teensy Header & 2x7 Header & Digikey & 609-3485-1-ND & 61001421121 \\
2 & For U5 & Teensy Header & 1x14 Header & Digikey & 732-5325-ND & 61301411121 \\
1 & For U5 & Teensy Socket & 2x7 Socket & Digikey & S7110-ND & PPPC072LFBN-RC \\
2 & For U5 & Teensy Socket & 1x14 Socket & Digikey & S7012-ND & PPTC141LFBN-RC \\
1 & For U6 & LCD Header & 1x16 Header & Digikey & 732-5327-ND & 61301611121 \\
1 & For U6 & LCD Socket & 1x16 Socket & Digikey & S7014-ND & PPTC161LFBN-RC \\
2 & J10, J11 & Stepper Stick Header & 1x8 Header & Digikey & 732-5321-ND & 61300811121 \\
2 & J10, J11 & Stepper Stick Socket & 1x8 Socket & Digikey & S7006-ND & PPTC081LFBN-RC \\
5 & J1, J8, J9 & N/A & 1x4 Socket & Digikey & S7002-ND & PPTC041LFBN-RC \\
3 & J3, J4, J5 & N/A & 1x4 SMD Header & Digikey & SAM12225-ND & TSM-104-02-L-SV \\
2 & For U2 & N/A & Board Spacers & Digikey & 952-3465-ND & R6010-00 \\
3 & M4, M5, M6 & N/A & M3 SMD Nuts & Digikey & 732-10918-1-ND & 9775056360R \\
1 & J2 & N/A & Screw Terminal & Digikey & ED10563-ND & OSTVN04A150 \\
1 & J1 & PJ-063BH & Barrel Jack & Digikey & CP-063BH-ND & PJ-063BH \\
4 & C2, C6, C7, C8 & 10 nF, 50 V & Capacitor, X7R & Digikey & 478-1227-1-ND & 06035C103KAT2A \\
1 & C16 & 470 nF, 10 V & Capacitor, X7R & Digikey & 1763-1190-1-ND & 0402BW474K160YT \\
2 & C1, C3 & 1 uF, 50 V & Capacitor, X7R & Digikey & 587-3247-1-ND & UMK107AB7105KA-T \\
1 & C4 & 15 uF & Tantalum Capacitor & Digikey & 399-3775-1-ND & T491D156K035AT \\
1 & C5 & 220 uF & Tantalum Capacitor & Digikey & 399-9744-1-ND & T491D227M010AT \\
1 & R6 & 470 \si{\ohm}& Resistor, Thick film & Digikey & Common & Common \\
2 & R1, R9, R10, R11, R12 & 4.7 k\si{\ohm} & Resistor, Thick film & Digikey & Common & Common \\
5 & R2, R3, R4, R5, R17 & 10 k\si{\ohm} & Resistor, Thick film & Digikey & Common & Common \\
1 & R13 & 20 k\si{\ohm} & Resistor, Thick film & Digikey & Common & Common \\
2 & R8, R16 & 20 k\si{\ohm} & Trim Pot & Digikey & SM-42TX203CT-ND & SM-42TX203 \\
1 & L1 & 68 uH & Inductor & Digikey & 513-1386-1-ND & DR125-680-R \\
1 & D1 & DB2W60400L & Schottky Diode & Digikey & P15184CT-ND & DB2W60400L \\
1 & U1 & IS31FL3193 & LED Driver & Digikey & 706-1215-1-ND & IS31FL3193-DLS2-TR \\
1 & U3 & MAX6816 & Debouncer & Digikey & MAX6816EUS+TCT-ND & MAX6816EUS+T \\
1 & U2 & MIC4680-5.0YM-TR & DC DC Converter & Digikey & 576-1221-1-ND & MIC4680-5.0YM-TR \\
1 & U4 & TMC2100 & Silent Stepper Stick & Digikey & 1460-1159-ND & N/A \\
1 & N/A & N/A & Enclosure & Digikey & 36-705-ND & 705 \\

\bottomrule
\end{tabular}
\end{table}








\medskip

\bibliographystyle{unsrt}
\bibliography{refs}

\end{document}
